\documentclass[11pt,a4paper]{article}
\usepackage{isabelle,isabellesym}
\usepackage{amssymb}
\usepackage[english]{babel}
\usepackage[only,bigsqcap]{stmaryrd}
\usepackage{wasysym}

% this should be the last package used
\usepackage{pdfsetup}

% urls in roman style, theory text in math-similar italics
\urlstyle{rm}
\isabellestyle{it}

% Tweaks
\newcounter{TTStweak_tag}
\setcounter{TTStweak_tag}{0}
\newcommand{\setTTS}{\setcounter{TTStweak_tag}{1}}
\newcommand{\resetTTS}{\setcounter{TTStweak_tag}{0}}
\newcommand{\insertTTS}{\ifnum\value{TTStweak_tag}=1 \ \ \ \fi}

\renewcommand{\isakeyword}[1]{\resetTTS\emph{\bf\def\isachardot{.}\def\isacharunderscore{\isacharunderscorekeyword}\def\isacharbraceleft{\{}\def\isacharbraceright{\}}#1}}
\renewcommand{\isachardoublequoteopen}{\insertTTS}
\renewcommand{\isachardoublequoteclose}{\setTTS}
\renewcommand{\isanewline}{\mbox{}\par\mbox{}\resetTTS}



\renewcommand{\isamarkupcmt}[1]{\hangindent5ex{\isastylecmt --- #1}}

%\newcommand{\isaheader}[1]{\section{#1}}
%\renewcommand{\isamarkupheader}[1]{#1}

\newcommand{\DefineSnippet}[2]{#2}

\begin{document}

\title{Formalizing the Edmonds-Karp Algorithm}
\author{Peter Lammich and S.~Reza Sefidgar}
\maketitle

\begin{abstract}
We present a formalization of the Ford-Fulkerson method for computing the maximum flow in a network.
Our formal proof closely follows a standard textbook proof, and is accessible even without being
an expert in Isabelle/HOL--- the interactive theorem prover used for the formalization.
We then use stepwise refinement to obtain the Edmonds-Karp algorithm, and formally prove a bound on its complexity.
Further refinement yields a verified implementation, whose execution time compares well to an unverified reference implementation in Java.

This entry is based on our ITP-2016 paper with the same title.
\end{abstract}

\clearpage
\tableofcontents

\clearpage

% sane default for proof documents
\parindent 0pt\parskip 0.5ex

\section{Introduction}
Computing the maximum flow of a network is an important problem in graph theory.
Many other problems, like maximum-bipartite-matching, edge-disjoint-paths,
circulation-demand, as well as various scheduling and resource allocating
problems can be reduced to it. The Ford-Fulkerson method~\cite{FF56} describes a
class of algorithms to solve the maximum flow problem. An important instance is
the Edmonds-Karp algorithm~\cite{EK72}, which was one of the first algorithms to
solve the maximum flow problem in polynomial time for the general case of
networks with real valued capacities.

In our paper~\cite{LaSe16}, we present a formal verification of the Edmonds-Karp algorithm
and its polynomial complexity bound. The formalization is conducted entirely in
the Isabelle/HOL proof assistant~\cite{NPW02}. This entry contains the complete formalization.
Stepwise refinement
techniques~\cite{Wirth71,Back78,BaWr98} allow us to elegantly structure our
verification into an abstract proof of the Ford-Fulkerson method, its
instantiation to the Edmonds-Karp algorithm, and finally an efficient
implementation. The abstract parts of our verification closely follow the
textbook presentation of Cormen et al.~\cite{CLRS09}. We have used the
Isar~\cite{Wenzel99} proof language to develop human-readable proofs that are 
accessible even to non-Isabelle experts.

While there exists another formalization of the Ford-Fulkerson method in
Mizar~\cite{Lee05}, we are, to the best of our knowledge, the first that verify a
polynomial maximum flow algorithm, prove the polynomial complexity bound, or
provide a verified executable implementation. Moreover, this entry is a case
study on elegantly formalizing algorithms.
% generated text of all theories
\input{session}

\section{Conclusion}\label{sec:concl}
  We have presented a verification of the Edmonds-Karp algorithm, using a
  stepwise refinement approach. Starting with a proof of the Ford-Fulkerson
  theorem, we have verified the generic Ford-Fulkerson method, specialized it to
  the Edmonds-Karp algorithm, and proved the upper bound $O(VE)$ for the number
  of outer loop iterations. We then conducted several refinement steps to derive
  an efficiently executable implementation of the algorithm, including a
  verified breadth first search algorithm to obtain shortest augmenting paths.
  Finally, we added a verified algorithm to check whether the input is a valid
  network, and generated executable code in SML. The runtime of our verified
  implementation compares well to that of an unverified reference implementation
  in Java. Our formalization has combined several techniques to achieve an
  elegant and accessible formalization: Using the Isar proof
  language~\cite{Wenzel99}, we were able to provide a completely rigorous but
  still accessible proof of the Ford-Fulkerson theorem. The Isabelle Refinement
  Framework~\cite{LaTu12,La12} and the Sepref tool~\cite{La15,La16} allowed us
  to present the Ford-Fulkerson method on a level of abstraction that closely
  resembles pseudocode presentations found in textbooks, and then formally link
  this presentation to an efficient implementation. Moreover, modularity of
  refinement allowed us to develop the breadth first search algorithm
  independently, and later link it to the main algorithm. The BFS algorithm can
  be reused as building block for other algorithms. The data structures are
  re-usable, too: although we had to implement the array representation of
  (capacity) matrices for this project, it will be added to the growing library
  of verified imperative data structures supported by the Sepref tool, such that
  it can be re-used for future formalizations. During this project, we have
  learned some lessons on verified algorithm development: 
  \begin{itemize} 
  \item
  It is important to keep the levels of abstraction strictly separated. For
  example, when implementing the capacity function with arrays, one needs to
  show that it is only applied to valid nodes. However, proving that, e.g.,
  augmenting paths only contain valid nodes is hard at this low level. Instead,
  one can protect the application of the capacity function by an assertion---
  already on a high abstraction level where it can be easily discharged. On
  refinement, this assertion is passed down, and ultimately available for the
  implementation. Optimally, one wraps the function together with an assertion
  of its precondition into a new constant, which is then refined independently.
  \item Profiling has helped a lot in identifying candidates for optimization.
  For example, based on profiling data, we decided to delay a possible
  deforestation optimization on augmenting paths, and to first refine the
  algorithm to operate on residual graphs directly. 
  \item ``Efficiency bugs''
  are as easy to introduce as for unverified software. For example, out of
  convenience, we implemented the successor list computation by \emph{filter}.
  Profiling then indicated a hot-spot on this function. As the order of
  successors does not matter, we invested a bit more work to make the
  computation tail recursive and gained a significant speed-up. Moreover, we
  realized only lately that we had accidentally implemented and verified
  matrices with column major ordering, which have a poor cache locality for our
  algorithm. Changing the order resulted in another significant speed-up.
  \end{itemize} 
  We conclude with some statistics: The formalization consists of
  roughly 8000 lines of proof text, where the graph theory up to the
  Ford-Fulkerson algorithm requires 3000 lines. The abstract Edmonds-Karp
  algorithm and its complexity analysis contribute 800 lines, and its
  implementation (including BFS) another 1700 lines. The remaining lines are
  contributed by the network checker and some auxiliary theories. The
  development of the theories required roughly 3 man month, a significant amount
  of this time going into a first, purely functional version of the
  implementation, which was later dropped in favor of the faster imperative
  version. 

  \subsection{Related Work}\label{sec:related_work} 
  We are only aware
  of one other formalization of the Ford-Fulkerson method conducted in
  Mizar~\cite{MaRu05} by Lee. Unfortunately, there seems to be no publication on
  this formalization except~\cite{Lee05}, which provides a Mizar proof script
  without any additional comments except that it ``defines and proves
  correctness of Ford/Fulkerson's Maximum Network-Flow algorithm at the level of
  graph manipulations''. Moreover, in Lee et al.~\cite{LeRu07}, which is about
  graph representation in Mizar, the formalization is shortly mentioned, and it
  is clarified that it does not provide any implementation or data structure
  formalization. As far as we understood the Mizar proof script, it formalizes
  an algorithm roughly equivalent to our abstract version of the Ford-Fulkerson
  method. Termination is only proved for integer valued capacities. Apart from
  our own work~\cite{La14,NoLa12}, there are several other verifications of
  graph algorithms and their implementations, using different techniques and
  proof assistants. Noschinski~\cite{Nosch15} verifies a checker for
  (non-)planarity certificates using a bottom-up approach. Starting at a C
  implementation, the AutoCorres tool~\cite{Greenaway15,GAK12} generates a
  monadic representation of the program in Isabelle. Further abstractions are
  applied to hide low-level details like pointer manipulations and fixed size
  integers. Finally, a verification condition generator is used to prove the
  abstracted program correct. Note that their approach takes the opposite
  direction than ours: While they start at a concrete version of the algorithm
  and use abstraction steps to eliminate implementation details, we start at an
  abstract version, and use concretization steps to introduce implementation
  details.

  Chargu\'eraud~\cite{char11} also uses a bottom-up approach to verify
  imperative programs written in a subset of OCaml, amongst them a version of
  Dijkstra's algorithm: A verification condition generator generates a
  \emph{characteristic formula}, which reflects the semantics of the program in
  the logic of the Coq proof assistant~\cite{BeCa10}. 

  \subsection{Future Work}
  Future work includes the optimization of our implementation, and the
  formalization of more advanced maximum flow algorithms, like Dinic's
  algorithm~\cite{Di06} or push-relabel algorithms~\cite{GoTa88}. We expect both
  formalizing the abstract theory and developing efficient implementations to be
  challenging but realistic tasks.
% optional bibliography
\bibliographystyle{abbrv}
\bibliography{root}

\end{document}

%%% Local Variables:
%%% mode: latex
%%% TeX-master: t
%%% End:
