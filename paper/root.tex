% Markierungen: TODO
\documentclass{llncs}

\usepackage{etex}
\pagestyle{plain} % turn on page numbers
\usepackage[utf8]{inputenc}
\usepackage{microtype} % Better typesetting for PDFs -- is enabling this ok?
\usepackage{amsmath}
\usepackage{amssymb}
%\usepackage{eufrak} %The eufrak package is redundant if the amsfonts package is used
% \usepackage{mathpartir}
%\DeclareMathAlphabet{\mathpzc}{OT1}{pzc}{m}{it}
\usepackage[boxed]{algorithm}
\usepackage{enumerate}
\usepackage{listings}
\usepackage{graphicx}
\usepackage{tabularx}
\usepackage{booktabs}
\usepackage{color}
\usepackage[noend]{algpseudocode}
\usepackage{caption}
\usepackage[font=scriptsize]{subcaption}
\usepackage{hyperref}
\usepackage{float}
\usepackage{wrapfig}
\usepackage{multirow}
\usepackage{pgfplots}


\usepackage{packages/isabelle}
\usepackage{packages/isabelletags}
\usepackage{packages/isabellesym}
\usepackage{packages/comment}

\def\isadelimproof{}
\def\endisadelimproof{}
\def\isatagproof{}
\def\endisatagproof{}
\def\isafoldproof{}
\def\isadelimproof{}
\def\endisadelimproof{}

% General
\newcommand{\ie}{i.\,e.\ }
\newcommand{\eg}{e.\,g.\ }
\newcommand{\wrt}{wrt.\ }
\newcommand{\cf}{cf.\ }
\newcommand{\etc}{etc.\ }


%Isabelle/HOL
\newcommand{\SOME}{\varepsilon}
\newcommand{\union}{\cup}
\newcommand{\inter}{\cap}
\renewcommand{\and}{\land}

\newcommand{\set}{{\sf set}}

\newcommand{\Sup}{\bigsqcup}
\newcommand{\sle}{\sqsubseteq}

\newcommand{\la}{{\sf \gets}}

\newcommand{\True}{{\sf True}}
\newcommand{\False}{{\sf False}}
\newcommand{\nat}{{\mathbb N}}
\newcommand{\bool}{{\mathbb B}}
\newcommand{\unit}{{\sf unit}}
\newcommand{\last}{{\sf last}}
\newcommand{\butlast}{{\sf butlast}}

\newcommand{\V}{V}

\renewcommand{\implies}{\Longrightarrow}

\newcommand{\isaconst}[1]{\textsl{#1}}
\newcommand{\isakeyw}[1]{\textsf{\bf #1}}
\newcommand{\isalem}[1]{\textsf{#1}}

\newcommand{\locale}{\isakeyw{locale}}

\newcommand{\lrec}{\mathopen{(\!|}}
\newcommand{\rrec}{\mathclose{|\!)}}

\newcommand{\lsem}{\ensuremath{\mathopen{[\![}}}
\newcommand{\rsem}{\ensuremath{\mathclose{]\!]}}}

\newcommand{\up}{\uparrow}

% Refinement Framework
\newcommand{\SUCCEED}{{\sf\bf succeed}}
\newcommand{\FAIL}{{\sf\bf fail}}
\newcommand{\fail}{{\sf\bf fail}}
\newcommand{\RES}{{\sf\bf res}}
\newcommand{\SPEC}{{\sf\bf spec}}
\newcommand{\return}{\RETURN}
\newcommand{\bind}{\BIND}
\newcommand{\RETURN}{{\sf\bf return}}
\newcommand{\BIND}{{\sf\bf bind}}
\newcommand{\assert}{\ASSERT}
\newcommand{\ASSERT}{{\sf\bf assert}}
\renewcommand{\do}{\DO}
\newcommand{\DO}{{\sf\bf do}}
\newcommand{\IN}{{\sf\bf in}}
\newcommand{\IF}{{\sf\bf if}}
\newcommand{\THEN}{{\sf\bf then}}
\newcommand{\ELSE}{{\sf\bf else}}
\newcommand{\LET}{{\sf\bf let}}
\newcommand{\WHILE}{{\sf\bf while}}
\newcommand{\WHILET}{\WHILE_{\sf\bf T}}
\newcommand{\FOREACH}{{\sf\bf foreach}}
\newcommand{\FOREACHC}{\FOREACH_{\sf\bf C}}
\newcommand{\REC}{{\sf\bf rec}}
\newcommand{\RECT}{{\sf\bf rec_T}}

\newcommand{\Up}{{\Uparrow}}
\newcommand{\Down}{{\Downarrow}}

\newcommand{\lfp}{{\sf lfp}}
\newcommand{\gfp}{{\sf gfp}}
\newcommand{\body}{{\sf B}}
\newcommand{\htrip}[3]{{\models\{#1\}~{#2}~\{#3\}}}

\newcommand{\mono}{{\sf mono}}
\newcommand{\wf}{{\sf wf}}

%LTL
\newcommand{\Prop}{\mathsf{Prop}}
\newcommand{\Next}{\mathord{\mathsf{X}}}
\newcommand{\Finally}{\mathord{\mathsf{F}}}
\newcommand{\Globally}{\mathord{\mathsf{G}}}
\renewcommand{\Until}{\mathbin{\mathsf{U}}}
\newcommand{\Release}{\mathbin{\mathsf{V}}}

%Automata
\newcommand{\GBA}{\mathord{\mathsf{GBA}}}
\newcommand{\LGBA}{\mathord{\mathsf{LGBA}}}

\newcommand{\QGBA}{{{Q\_GBA}}}
\newcommand{\DGBA}{{{\Delta\_GBA}}}
\newcommand{\IGBA}{{{I\_GBA}}}
\newcommand{\FGBA}{{{F\_GBA}}}
\newcommand{\LLGBA}{{{L}}}

\newcommand{\BAtrans}{\mathsf{trans}}
\newcommand{\BAinitial}{\mathsf{initial}}
\newcommand{\BAaccept}{\mathsf{accept}}

%Misc
\newcommand{\equivalent}{\leftrightarrow}

\newcommand{\expand}{\mathbin{\mathsf{expand}}}
\newcommand{\expandbody}{\mathbin{\mathsf{expand\_body}}}
\def\by#1{\mathop{{\hbox{\setbox0=\hbox{$\scriptstyle{#1\quad}$}{$\buildrel{\>#1\>}\over{\hbox to \wd0{\rightarrowfill}}$}}}}}

%personalized comments
\newcommand{\as}[1]{\textcolor{blue}{\sf AS: #1}}
\newcommand{\AS}[1]{\marginpar{\footnotesize\textcolor{blue}{\sf AS: #1}}}
\newcommand{\RN}[1]{\marginpar{\footnotesize\textcolor{green}{\sf RN: #1}}}
\newcommand{\JE}[1]{\marginpar{\footnotesize\textcolor{green}{\sf JE: #1}}}
\newcommand{\PL}[1]{\marginpar{\footnotesize\textcolor{red}{\sf PL: #1}}}
\newcommand{\jgs}[1]{\textcolor{magenta}{\sf jgs: #1}}
\newcommand{\JGS}[1]{\marginpar{\footnotesize\textcolor{magenta}{\sf JGS: #1}}}



% Not really meant for highlighting isabelle source, but for easily writing latex that looks like
% isabelle
% 
% keyword level 1 - isabelle outer syntax 
% keyword level 2 - isabelle inner syntax programming constructs (if, let, etc)
% keyword level 3 - standard constants (length, mod, etc)
% keyword level 4 - isabelle proof methods

\lstdefinelanguage{isabelle}{
  morekeywords={theorem,theorems,corollary,lemma,lemmas,locale,begin,end,fixes,assumes,shows,
    constrains , definition, where, apply, done,unfolding, primrec, using, by, for, uses,
    schematic_lemma, concrete_definition, prepare_code_thms, export_code, datatype,
    proof, next, qed, show, have, hence, thus, interpretation, fix, context
 } ,
  morekeywords=[2]{rec, return, bind, foreach, if, then, else, and, do, let, in, res, spec, fail, assert, while, case, of},
%  morekeywords=[3]{length,mod,insert},
%   morekeywords=[4]{simp,auto,intro,elim,rprems,refine_mono,refine_rcg},
  sensitive=True,
  morecomment=[s]{(\*}{\*)},
}

\lstset{
    language=isabelle,
    mathescape=true,
    escapeinside={--"}{"},
    basicstyle={\itshape},
    keywordstyle=\rm\bfseries,
    keywordstyle=[2]\rm\tt,
    keywordstyle=[3]\rm,
    keywordstyle=[4]\rm,
    showstringspaces=false,
    keepspaces=true,
    columns=[c]fullflexible}
\lstset{literate=
  {"}{}0
  {'}{{${}^\prime$}}1
  {\%}{{$\lambda$}}1
  {\\\%}{{$\lambda$}}1
  {\\\$}{{$\mathbin{\,\$\,}$}}1
  {->}{{$\rightarrow$}}1
  {<-}{{$\leftarrow$}}1
  {<.}{{$\langle$}}1
  {.>}{{$\rangle$}}1
  {<=}{{$\le$}}1
  {<->}{{$\leftrightarrow$}}1
  {-->}{{$\longrightarrow$}}2
  {<-->}{{$\longleftrightarrow$}}1
  {=>}{{$\Rightarrow$}}1
  {==}{{$\equiv$}}2
  {==>}{{$\implies$}}2
  {<=>}{{$\Leftrightarrow$}}1
  {~=}{{$\ne$}}1
  {!!}{{$\bigwedge$}}1
  {(}{{$($}}1
  {)}{{$)$}}1
  {\{}{{$\{$}}1
  {\}}{{$\}$}}1
  {[}{{$[$}}1
  {]}{{$]$}}1
  {(|}{{$\lrec$}}1
  {|)}{{$\rrec$}}1
  {[|}{{$\lsem$}}1
  {|]}{{$\rsem$}}1
  {\\<lbrakk>}{{$\lsem$}}1
  {\\<rbrakk>}{{$\rsem$}}1
  {|-}{{$\vdash$}}1
  {|->}{{$\mapsto$}}1
  {|_|}{{$\bigsqcup$}}1
  {...}{{$\dots$}}1
  {\\x}{{$\times$}}1
  {_0}{{${}_0$}}1
  {_1}{{${}_1$}}1
  {_2}{{${}_2$}}1
  {_3}{{${}_3$}}1
  {_4}{{${}_4$}}1
  {_5}{{${}_5$}}1
  {_6}{{${}_6$}}1
  {_7}{{${}_7$}}1
  {_8}{{${}_8$}}1
  {_9}{{${}_9$}}1
  {^*}{{$^*$}}1
  {\\<^sup>*}{{$^*$}}1
  {\\<^sub>*}{{$_*$}}1
  {\\<^sub>A}{{$_A$}}1
  {\\<^sub>r}{{$_r$}}1
  {\\<^sub>a}{{$_a$}}1
  {:_i}{{$:_i$}}1
  {\\<A>}{{$\mathcal{A}$}}1
  {\\<O>}{{\sf o}}1
  {\\<Phi>}{{$\Phi$}}1
  {\\<Psi>}{{$\Psi$}}1
  {\\<sigma>}{{$\sigma$}}1
  {\\<in>}{{$\in$}}1
  {\\<le>}{{$\le$}}1
  {\\<noteq>}{{$\ne$}}1
  {\\<lambda>}{{$\lambda$}}1
  {\\<longrightarrow>}{{$\longrightarrow$}}1
  {\\<longleftrightarrow>}{{$\longleftrightarrow$}}1
  {\\<Rightarrow>}{{$\Rightarrow$}}1
  {\\<Longrightarrow>}{{$\Longrightarrow$}}1
  {\\<rightarrow>}{{$\rightarrow$}}1
  {\\<leftarrow>}{{$\leftarrow$}}1
  {\\<mapsto>}{{$\mapsto$}}1
  {\\<equiv>}{{$\equiv$}}1
  {\\<and>}{{$\and$}}1
  {\\<or>}{{$\vee$}}1
  {\\<And>}{{$\bigwedge$}}1
  {\\<Up>}{{$\Uparrow$}}1
  {\\<Down>}{{$\Downarrow$}}1
  {\\<up>}{{$\uparrow$}}1
  {\\<down>}{{$\downarrow$}}1
  {\\<times>}{{$\times$}}1
  {\\<forall>}{{$\forall$}}1
  {\\<exists>}{{$\exists$}}1
  {\\<union>}{{$\cup$}}1
  {\\<Union>}{{$\bigcup$}}1
  {\\<inter>}{{$\cap$}}1
  {\\<subset>}{{$\subset$}}1
  {\\<subseteq>}{{$\subseteq$}}1
  {\\<supset>}{{$\supset$}}1
  {\\<supseteq>}{{$\supseteq$}}1
  {\\<alpha>}{{$\alpha$}}1
  {\\<beta>}{{$\beta$}}1
  {\\<gamma>}{{$\gamma$}}1
  {\\alpha}{{$\alpha$}}1
  {\\beta}{{$\beta$}}1
  {\\gamma}{{$\gamma$}}1
  {\\<Gamma>}{{$\Gamma$}}1
  {\\<langle>}{{$\langle$}}1
  {\\<rangle>}{{$\rangle$}}1
  {\\<not>}{{$\neg$}}1
  {\\<notin>}{{$\notin$}}1
  {\\<guillemotright>}{{$\gg$}}1
  {\\in}{$\in$}1
  {\\and}{$\wedge$}1
  {\\or}{$\vee$}1
  {\\Phi}{{$\Phi$}}1
  {\\Psi}{{$\Psi$}}1
  {\\le}{{$\le$}}1
  {\\Up}{{$\Uparrow$}}1
  {\\Down}{{$\Down$}}1
  {>>}{{$\gg$}}1
  {>>=}{{${\gg}{=}$}}1
  {<*lex*>}{{$\times_{\sf lex}$}}1
}

\newcommand{\isai}{\lstinline[language=isabelle]}

\lstset{captionpos=b}
\lstset{numberbychapter=false}

% Include snippets
\newcommand{\DefineSnippet}[2]{%
   \expandafter\newcommand\csname snippet--#1\endcsname{%
     \begin{quote}
     \begin{isabelle}\footnotesize
     #2
     \end{isabelle}
     \end{quote}}}
\newcommand{\Snippet}[1]{\csname snippet--#1\endcsname}
\DefineSnippet{augment_flow_presv_cap}{
\isacommand{lemma}\isamarkupfalse%
\ augment{\isacharunderscore}flow{\isacharunderscore}presv{\isacharunderscore}cap{\isacharcolon}\ \isanewline
\ \ \isakeyword{shows}\ {\isachardoublequoteopen}{\isadigit{0}}\ {\isasymle}\ augment\ f{\isacharprime}{\isacharparenleft}u{\isacharcomma}v{\isacharparenright}\ {\isasymand}\ augment\ f{\isacharprime}{\isacharparenleft}u{\isacharcomma}v{\isacharparenright}\ {\isasymle}\ c{\isacharparenleft}u{\isacharcomma}v{\isacharparenright}{\isachardoublequoteclose}\isanewline
%
\isadelimproof
%
\endisadelimproof
%
\isatagproof
\isacommand{proof}\isamarkupfalse%
\ {\isacharparenleft}cases\ {\isachardoublequoteopen}{\isacharparenleft}u{\isacharcomma}v{\isacharparenright}{\isasymin}E{\isachardoublequoteclose}{\isacharsemicolon}\ rule\ conjI{\isacharparenright}\ \isacommand{case}\isamarkupfalse%
\ True\isanewline
\ \ \isacommand{have}\isamarkupfalse%
\ {\isachardoublequoteopen}f{\isacharprime}{\isacharparenleft}v{\isacharcomma}u{\isacharparenright}\ {\isasymle}\ cf{\isacharparenleft}v{\isacharcomma}u{\isacharparenright}{\isachardoublequoteclose}\ \isacommand{using}\isamarkupfalse%
\ f{\isacharprime}{\isachardot}capacity{\isacharunderscore}const\ \isacommand{by}\isamarkupfalse%
\ auto\isanewline
\ \ \isacommand{also}\isamarkupfalse%
\ \isacommand{from}\isamarkupfalse%
\ {\isacartoucheopen}{\isacharparenleft}u{\isacharcomma}v{\isacharparenright}{\isasymin}E{\isacartoucheclose}\ \isacommand{have}\isamarkupfalse%
\ {\isachardoublequoteopen}cf{\isacharparenleft}v{\isacharcomma}u{\isacharparenright}\ {\isacharequal}\ f{\isacharparenleft}u{\isacharcomma}v{\isacharparenright}{\isachardoublequoteclose}\isanewline
\ \ \ \ \isacommand{using}\isamarkupfalse%
\ no{\isacharunderscore}parallel{\isacharunderscore}edge\ \isacommand{by}\isamarkupfalse%
\ {\isacharparenleft}auto\ simp{\isacharcolon}\ residualGraph{\isacharunderscore}def{\isacharparenright}\isanewline
\ \ \isacommand{finally}\isamarkupfalse%
\ \isacommand{have}\isamarkupfalse%
\ {\isachardoublequoteopen}f{\isacharprime}{\isacharparenleft}v{\isacharcomma}u{\isacharparenright}\ {\isasymle}\ f{\isacharparenleft}u{\isacharcomma}v{\isacharparenright}{\isachardoublequoteclose}\ \isacommand{{\isachardot}}\isamarkupfalse%
\isanewline
\ \ \isacommand{have}\isamarkupfalse%
\ {\isachardoublequoteopen}augment\ f{\isacharprime}{\isacharparenleft}u{\isacharcomma}v{\isacharparenright}\ {\isacharequal}\ f{\isacharparenleft}u{\isacharcomma}v{\isacharparenright}\ {\isacharplus}\ f{\isacharprime}{\isacharparenleft}u{\isacharcomma}v{\isacharparenright}\ {\isacharminus}\ f{\isacharprime}{\isacharparenleft}v{\isacharcomma}u{\isacharparenright}{\isachardoublequoteclose}\isanewline
\ \ \ \ \isacommand{using}\isamarkupfalse%
\ {\isacartoucheopen}{\isacharparenleft}u{\isacharcomma}v{\isacharparenright}{\isasymin}E{\isacartoucheclose}\ \isacommand{by}\isamarkupfalse%
\ {\isacharparenleft}auto\ simp{\isacharcolon}\ augment{\isacharunderscore}def{\isacharparenright}\isanewline
\ \ \isacommand{also}\isamarkupfalse%
\ \isacommand{have}\isamarkupfalse%
\ {\isachardoublequoteopen}{\isasymdots}\ {\isasymge}\ f{\isacharparenleft}u{\isacharcomma}v{\isacharparenright}\ {\isacharplus}\ f{\isacharprime}{\isacharparenleft}u{\isacharcomma}v{\isacharparenright}\ {\isacharminus}\ f{\isacharparenleft}u{\isacharcomma}v{\isacharparenright}{\isachardoublequoteclose}\isanewline
\ \ \isacommand{using}\isamarkupfalse%
\ {\isacartoucheopen}f{\isacharprime}{\isacharparenleft}v{\isacharcomma}u{\isacharparenright}\ {\isasymle}\ f{\isacharparenleft}u{\isacharcomma}v{\isacharparenright}{\isacartoucheclose}\ \isacommand{by}\isamarkupfalse%
\ auto\isanewline
\ \ \isacommand{also}\isamarkupfalse%
\ \isacommand{have}\isamarkupfalse%
\ {\isachardoublequoteopen}{\isasymdots}\ {\isacharequal}\ f{\isacharprime}{\isacharparenleft}u{\isacharcomma}v{\isacharparenright}{\isachardoublequoteclose}\ \isacommand{by}\isamarkupfalse%
\ auto\isanewline
\ \ \isacommand{also}\isamarkupfalse%
\ \isacommand{have}\isamarkupfalse%
\ {\isachardoublequoteopen}{\isasymdots}\ {\isasymge}\ {\isadigit{0}}{\isachardoublequoteclose}\ \isacommand{using}\isamarkupfalse%
\ f{\isacharprime}{\isachardot}capacity{\isacharunderscore}const\ \isacommand{by}\isamarkupfalse%
\ auto\isanewline
\ \ \isacommand{finally}\isamarkupfalse%
\ \isacommand{show}\isamarkupfalse%
\ {\isachardoublequoteopen}augment\ f{\isacharprime}{\isacharparenleft}u{\isacharcomma}v{\isacharparenright}\ {\isasymge}\ {\isadigit{0}}{\isachardoublequoteclose}\ \isacommand{{\isachardot}}\isamarkupfalse%
\isanewline
\ \ \ \ \ \ \isanewline
\ \ \isacommand{have}\isamarkupfalse%
\ {\isachardoublequoteopen}augment\ f{\isacharprime}{\isacharparenleft}u{\isacharcomma}v{\isacharparenright}\ {\isacharequal}\ f{\isacharparenleft}u{\isacharcomma}v{\isacharparenright}\ {\isacharplus}\ f{\isacharprime}{\isacharparenleft}u{\isacharcomma}v{\isacharparenright}\ {\isacharminus}\ f{\isacharprime}{\isacharparenleft}v{\isacharcomma}u{\isacharparenright}{\isachardoublequoteclose}\isanewline
\ \ \ \ \isacommand{using}\isamarkupfalse%
\ {\isacartoucheopen}{\isacharparenleft}u{\isacharcomma}v{\isacharparenright}{\isasymin}E{\isacartoucheclose}\ \isacommand{by}\isamarkupfalse%
\ {\isacharparenleft}auto\ simp{\isacharcolon}\ augment{\isacharunderscore}def{\isacharparenright}\isanewline
\ \ \isacommand{also}\isamarkupfalse%
\ \isacommand{have}\isamarkupfalse%
\ {\isachardoublequoteopen}{\isasymdots}\ {\isasymle}\ f{\isacharparenleft}u{\isacharcomma}v{\isacharparenright}\ {\isacharplus}\ f{\isacharprime}{\isacharparenleft}u{\isacharcomma}v{\isacharparenright}{\isachardoublequoteclose}\ \isacommand{using}\isamarkupfalse%
\ f{\isacharprime}{\isachardot}capacity{\isacharunderscore}const\ \isacommand{by}\isamarkupfalse%
\ auto\isanewline
\ \ \isacommand{also}\isamarkupfalse%
\ \isacommand{have}\isamarkupfalse%
\ {\isachardoublequoteopen}{\isasymdots}\ {\isasymle}\ f{\isacharparenleft}u{\isacharcomma}v{\isacharparenright}\ {\isacharplus}\ cf{\isacharparenleft}u{\isacharcomma}v{\isacharparenright}{\isachardoublequoteclose}\ \isacommand{using}\isamarkupfalse%
\ f{\isacharprime}{\isachardot}capacity{\isacharunderscore}const\ \isacommand{by}\isamarkupfalse%
\ auto\isanewline
\ \ \isacommand{also}\isamarkupfalse%
\ \isacommand{have}\isamarkupfalse%
\ {\isachardoublequoteopen}{\isasymdots}\ {\isacharequal}\ f{\isacharparenleft}u{\isacharcomma}v{\isacharparenright}\ {\isacharplus}\ c{\isacharparenleft}u{\isacharcomma}v{\isacharparenright}\ {\isacharminus}\ f{\isacharparenleft}u{\isacharcomma}v{\isacharparenright}{\isachardoublequoteclose}\isanewline
\ \ \ \ \isacommand{using}\isamarkupfalse%
\ {\isacartoucheopen}{\isacharparenleft}u{\isacharcomma}v{\isacharparenright}{\isasymin}E{\isacartoucheclose}\ \isacommand{by}\isamarkupfalse%
\ {\isacharparenleft}auto\ simp{\isacharcolon}\ residualGraph{\isacharunderscore}def{\isacharparenright}\isanewline
\ \ \isacommand{also}\isamarkupfalse%
\ \isacommand{have}\isamarkupfalse%
\ {\isachardoublequoteopen}{\isasymdots}\ {\isacharequal}\ c{\isacharparenleft}u{\isacharcomma}v{\isacharparenright}{\isachardoublequoteclose}\ \isacommand{by}\isamarkupfalse%
\ auto\isanewline
\ \ \isacommand{finally}\isamarkupfalse%
\ \isacommand{show}\isamarkupfalse%
\ {\isachardoublequoteopen}augment\ f{\isacharprime}{\isacharparenleft}u{\isacharcomma}\ v{\isacharparenright}\ {\isasymle}\ c{\isacharparenleft}u{\isacharcomma}\ v{\isacharparenright}{\isachardoublequoteclose}\ \isacommand{{\isachardot}}\isamarkupfalse%
\isanewline
\isacommand{qed}\isamarkupfalse%
\ {\isacharparenleft}auto\ simp{\isacharcolon}\ augment{\isacharunderscore}def\ cap{\isacharunderscore}positive{\isacharparenright}%
\endisatagproof
{\isafoldproof}%
%
\isadelimproof
%
\endisadelimproof
%
}%EndSnippet



% \overfullrule=8pt

\begin{document}

\title{Formalizing the Edmonds-Karp Algorithm}
% \subtitle{}

\author{S.~Reza Sefidgar, Peter Lammich}

\institute{Technische Universit\"at M\"unchen, \email{\{sefidgar,lammich\}@in.tum.de}}

\maketitle
\begin{abstract}
We present a formalization of the Ford-Fulkerson method for computing the maximum flow in a network.
Our formal proof closely follows a standard textbook proof, and is easy to read even without being
an expert in Isabelle/HOL--- the interactive theorem prover used for the formalization.
% 
% 
% so it is easy to read even for those who 
% are not expert in Isabelle/HOL--- the interactive theorem prover used for the formalization.

We then use stepwise refinement to obtain the Edmonds-Karp algorithm, and formally prove a bound on its complexity.
Further refinement yields a verified implementation, which compares well to an unverified reference implementation in Java.

%We present a formalization of the Ford-Fulkerson method for computing the maximum flow in a network. 
%The formal proofs are done in Isabelle/HOL. They closely follow a standard textbook proof and are relatively easy 
%to read, even without being an Isabelle expert.

%We then use stepwise refinement to obtain the Edmonds-Karp algorithm and prove its upper complexity bound of $O(|V||E|^2)$.
%Further refinement yields a verified implementation, which is almost as fast as an unverified  
%reference implementation in Java. 

% 
% and a verified imperative implementation of it. Our implementation is roughly 2.5 times slower than an unverified  
% reference implementation in Java. 
% 
% Finally, we prove the upper complexity bound of $O(|V||E|^2)$ for the Edmonds-Karp algorithm.
\end{abstract}

\section{Introduction}
The Ford-Fulkerson algorithm~\cite{FF56} is one of the important results of the graph theory and is used to find a solution to the maximum flow problem in flow networks. Many important mathematical problems like the maximum-bipartite-matching problem, the edge-disjoint-paths problem, the circulation-demand problem, and many other scheduling and resource allocating problems can be reduced to the maximum flow problem. Hence, the Ford-Fulkerson algorithm has major application in the field of mathematical optimization.

Despite its importance, no formalization of the Ford-Fulkerson algorithm has been developed in modern proof assistants. The only similar work that the authors are aware of is formalization of the Ford-Fulkerson algorithm in Mizar~\cite{Lee05}. This formalization defines and proves correctness of the algorithm at the level of graph manipulations without providing concrete implementation of the algorithm. Providing such an implementation is specially important, as it could provide the basis for many verfied programs with practical importance.

(TODO: Here I have mentioned the Edmonds-Karp for the first time.) This paper present a formalization of the correctness proof of the Ford-Fulkerson method in Isabelle/HOL~\cite{NPW02}. Our formalization is based on the informal proof presented in the book \emph{Introduction to algorithms}~\cite{CLRS09}. We then provide a verified implementation of the Edmons-Karp's algorithm~\cite{EK72}. This algorithm instantiates the Ford-Fulkerson method with breadth-first-search and provides a polynomial algorithm for solving the maximum flow problem in generic case--- maximum flow on networks with real edge capacities.

Isabelle/HOL provides some automation for generating the code corresponding to a verified algorithm, however, in order to use the code generating features, one needs to do the formalization with executability in mind. Being limited only to executable concepts in the formalization has the disadvantage of cluttering the proofs with implementation details. Such approach makes the proofs more complicated, and may even render proofs of medium complex algorithms unmanageable.

(TODO: I am not sure about the level of details that I describe refinement in introduction, co-relate this with the refinement section) One solution for the aforementioned problem is refinement~\cite{Wirth71}. In order to generate executable code of an algorithm using refinement, we first formulate the algorithm on an abstract level. The abstract version of the algorithm has a clean correctness proof as it only captures the idea behind the algorithm. Next, we refine the abstract definition of the algorithm towards an executable implementation in possibly multiple refinement steps. During each step, we only need to prove the correctness of the implementation of a particular abstract concept. Hence, we can be sure about the correctness of the resulting executable program as each refinement step preserves the correctness.

(TODO: check the citation order and description) There are several approaches to data refinement in Isabelle/HOL. We will be using the Isabelle Refinement Framework~\cite{LaTu12,La12}, which has been used for proving more complex results such as formalized implementation of Hopcroft’s DFA-minimization algorithm~\cite{LaTu12}. Given an algorithm phrased over abstract concepts like sets and maps, it automatically synthesizes a concrete, executable algorithm and the corresponding refinement theorem. To make it applicable for the development of actual algorithms, Autoref is integrated with the Isabelle Refinement Framework~\cite{La12} and the Isabelle Collection Framework~\cite{LL10}.


\section{A short background on the Ford-Fulkerson method}
% \begin{comment}
%   GOAL: 
%     1) Remind the reader of FoFu. The educated reader should have the feeling:
%       Yes, now I recall FoFu, and understand (again) what it does.
%       
%     2) Set the field for further, more detailed descriptions in next section.
%   
% 
%   Short description of Network: Finite graph over nodes V, 
%     edge (u,v) annotated with capacity c(u,v).
%     Assuming distinct nodes s and t. 
%     Note on additional assumptions: 
%         No parallel edges, s only outgoing, t only incoming, all nodes on path from s to t. 
%         Can transform any Network to match these assumptions, preserving the flow \cite{???}.
%         
%   s-t Flow: Annotation of edges with values, such that: 
%     1) Values smaller than capacities.
%     2) Kirchhoff law: sum of incoming flows = sum of outgoing flows, for all nodes except s t
%   
%   Value of flow: Incoming flow - outgoing flow of s
%   
%   Max-Flow problem. 
%     Rpt. why important?
%   
%   Cuts, intuition of min-cut >= max-flow
%   
%   Theorem: Min-Cut = Max-Flow
%   Proven via augmenting flow in residual graph. 
%     Present 1,2,3 of min-cut max-flow equivalences
%   
%   This yields an algorithm to compute max-flows, if we can compute flows in residual graph.
%     --> Simple way: Augmenting flow via augmenting path. ==> FoFu-scheme. 
%       Termination? In General: Only for integer capacities.
%       Edmonds-Karp: Shortest augmenting path. Always terminates! Complexity: O(|E||V|) outer loop iterations, BFS requires O|E|.
%   \end{comment}
A flow network is a weighted digraph where the set of vertices (nodes) $V$ is finite, and each edge $(u, v)$ has a real capacity $c (u, v) \geq 0$. There are two distinct vertices $s$ (source) and $t$ (sink) in the flow network. In this case the network is called an $s$-$t$ network. To make the formalization simple, some additional assumptions are added to the definition as following: 1) The source only has outgoing edges while the sink only has incoming edges; 2) If the flow network contains an edge $(u, v)$ then there is no edge $(v, u)$ in the reverse direction; and 3) every vertex of the flow network must be on a path from $s$ to $t$. Although these assumptions seems restrictive, any network may be transformed such that it satisfies all the aforementioned properties~\cite{CLRS09}.

An $s$-$t$ flow on a weighted digraph is a function $\colon V \times V \rightarrow \mathbb{R}$ which satisfies the following conditions. 1) \emph{Capacity constraint}: the flow of each edge is a non-negative value which is smaller or equal to the edge capacity; 2) \emph{Conservation constraint}: For all vertices except $s$ and $t$, the sum of flows over incoming edges of the vertex is equal to the sum of flows over outgoing edges of the vertex.

The value of an $s$-$t$ flow f is denoted by $|f|$, and is computed by subtracting the sum of the flows over all incoming edges of $s$ from the sum of the flows over all outgoing edges of $s$. Given an $s$-$t$ network $G$, the maximum flow problem involves finding a flow $f$ in $G$ such that the value of $f$ is greater or equal to the value of any other $s$-$t$ flow in $G$~\footnote{Note that the value of the flow in our model of a flow network equals sum of flows over edges going out of $s$. Because the model prohibits edges coming into the source.}.

One of the important results in mathematical optimization is the correlation between flows and cuts in flow networks. An $s$-$t$ cut in a flow network with source $s$ and sink $t$ is a partition of vertices that puts $s$ and $t$ in different subsets. Then the capacity of a cut is defined as sum of the edge capacities over all edges going from the source's side to the sink's side. Consider a valid flow in a flow network $G$. For any cut in $G$, the flow that crosses between the two subsets of the cut cannot exceed the capacity of the cut. So cuts define an upper bound for any flow in the flow network. The Ford-Fulkerson theorem tightens this bound and states that the value of the maximum flow is equal to the capacity of the minimum cut.

The Ford-Fulkerson algorithm is the direct consequence of the Ford-Fulkerson theorem. It solves the maximum flow problem in a greedy approach. Assume an instance of the maximum flow problem in an $s$-$t$ network. The algorithm starts with a zero flow: $f(e) = 0$ for all the edges $e$. In each iteration of the algorithm the value of $f$ is increased by pushing flow along a path from $s$ to $t$ up to the limits imposed by the available edge capacities. The process would continue until no more paths can be found to push additional flow from $s$ to $t$.

The total flow value is increased by manipulating the flow values along different edges. However, the flow along specific edges might decrease during this process. In order to perform these operations, the Ford-Fulkerson algorithm defines the residual graph of the flow network. Intuitively, the residual graph corresponding to a flow in a given flow network, has the same vertices as the flow network and consists of edges with capacities representing how flow can be changed along the edges of the flow network.

%Assume a flow $f$ in an $s$-$t$ network with capacity function $c$. An edge $(u, v)$ of the network can admit an additional amount of flow equal to the edge’s capacity minus the flow along that edge. If this value is positive, then a $(u, v)$ edge is added to the residual graph with a residual capacity $c_f (u, v) = c (u, v) - f (u, v)$. So, the residual graph only contains those edges of the original flow network that have left-over capacities. The residual graph may also contain some edges that do not exist in the original flow network. For each edge $(u, v)$ in the flow network which has a positive flow value, a $(v, u)$ edge is added to the residual graph with the residual capacity $c_f (v, u) = f (u, v)$. The edge $(v, u)$ indicates the possibility for pushing the flow backward along the edge $(u, v)$ in the original flow network.

(TODO: there is a shorter informal description available as comment. Adapt that one if required) There are two kinds of edges in the residual graph: 1) those of the residual graph that present the edges of the flow network with left-over capacities; and 2) those of the residual graph that do not occur in the flow network, and indicate the possibility for pushing the flow backward. Formally speaking, consider a flow $f$ on an $s$-$t$ network $G = (V, E)$ with the capacity function $c$. A residual capacity function $c_f \colon V \times V \rightarrow \mathbb{R}$ is defined as following:
\[ c_f (u, v) = 
	\begin{cases}
	c (u, v) - f(u, v) \hfill & \text{ if $(u, v) \in E$} \\
	f (v, u) \hfill & \text{ if $(v, u) \in E$} \\	
	0 \hfill & \text{ otherwise} \\
	\end{cases} 
\]

Then the residual graph corresponding to $f$ and $G$ is indicated by $G_f = (V, E_f)$, where the residual edge set $E_f$ is defined as following:
\[ E_f = \{(u,v) \in V \times V \colon c_f (u, v) > 0\}\]

After each flow update, the algorithm considers the residual graph corresponding to the new flow. Then the algorithm looks for an augmenting path--- a simple path from source to the sink--- in the residual graph. If such a path could be found, it will be used to construct a flow in the residual graph. Let $p$ be such a path. The residual capacity of $p$ is defined as $c_f(p) = min \{c_f(u, v): \text{$(u, v)$ is on  $p$}\}$. Then an augmenting flow $f_p \colon V \times V \rightarrow \mathbb{R}$  in the residual graph can be defined using $p$ as following:
\[ f_p (u, v) = 
	\begin{cases}
	c_f(p) \hfill & \text{ if $(u, v)$ is on $p$} \\	
	0 \hfill & \text{ otherwise} \\
	\end{cases} 
\]

This flow provides a road map for modifying the current flow in the flow network. The process of increasing the current flow using an augmenting flow is called augmentation. Let $f$ be the current flow in an $s$-$t$ network $G = (V, E)$ with capacity function $c$, and $f'$ be a flow in the corresponding residual graph. The augmentation of $f$ using $f'$ is a function from $V \times V$ to $\mathbb{R}$, defined as following:
\[ (f \uparrow f') (u, v) = 
	\begin{cases}
	f (u, v) + f'(u, v) - f'(v, u) \hfill & \text{ if $(u, v) \in E$} \\	
	0 \hfill & \text{ otherwise} \\
	\end{cases} 
\]

The intuition behind the above definition is based on the definition of the residual graph. The augment procedure increases the flow on $(u, v)$ by $f ( u, v )$ but decreases it by $f (v, u)$. This is due to the fact that pushing flow on the reverse edge on the residual graph signifies decreasing the flow in the original network. 

In each iteration of the algorithm we replace the current flow with the result of the augment function--- another flow on the network but with greater value. The algorithm terminates if no more augmenting paths can be found in the residual graph. According to the Ford-Fulkerson theorem, the flow computed after termination is indeed the maximum flow in the flow network. According to the theorem, the following statements are equivalent:

\begin{itemize}
\item $f$ is a maximum $s$-$t$ flow in flow network $G$.
\item there is no augmenting path in the residual graph $G_f$.
\item there is an $s$-$t$ cut $C$ in $G$ such that capacity of $C$ is equal to the value of $f$.
\end{itemize}

The termination of the Ford-Fulkerson algorithm depends on how the augmenting path is found. For the networks with irrational edge capacities the algorithm might even fail to terminate~\cite{Zwick95}. In practice, the maximum-flow problem often arises with integral capacities. Moreover, we may convert rational edge capacities to integral values by multiplying them with a big enough integer. If $f^*$ denotes a maximum-flow in the transformed network, then the algorithm will iterate at most $|f^*|$ times, since the augmentation procedure increases the flow value by at least one unit in each iteration. Assuming that we use breadth-first-search (BFS) or depth-first-search (DFS) for finding augmenting paths makes the total execution time of the Ford-Fulkerson algorithm $O ( E | f^* |)$, as BFS and DFS run in $O (E + V)$.

(TODO: Here I mention the Edmonds-Karp algorithm and I mention its running time, co-relate this with definition in later sections) The Ford-Fulkerson algorithm is sometimes called a method instead of an algorithm because it does not fully specify the procedure for finding the augmenting path in the residual graph. There are several implementations with different execution times. For instance, the Edmond-Karp algorithm uses breadth-first-search (BFS) for finding augmenting paths and has $O (VE^2)$ running-time. The augmentings paths that are computed using BFS are also shortest paths connecting source to the sink in the residual graph. Edmond and Karp showed that in the Ford-Fulkerson algorithm, if each augmenting path is the shortest one, the length of the augmenting paths is non-decreasing and the algorithm always terminates. Using this statement they could finally present a polynomial algorithm for solving the maximum flow problem in generic case of real edge capacities.


\section{Formalizing the Ford-Fulkerson method}
%  GOAL: Present highlights of our formalization, persuade the reader that we have done something substantial.
%    Reader should think: Yeah, they did cool stuff!
%
%  [???* Thematize definition of flows: On Graphs vs. on Networks. You find both in the literature. 
%      We decides for [...]. This is better suited, as is gives nice and elegant proofs, even formally, bla bla bla]
%    
%  Proof of MinCut-MaxFlow: Follows the textbook proof.
%    Uses Isar to even look like a textbook proof, being comprehensible even without running
%      Isabelle/HOL.
%      
%    For example, \isai{augment_flow_presv_cap}.
%      Present formal proof text. Perhaps oppose it to textbook proof?.
%    
%  Abstract Algo looks like pseudocode presented in texbooks.
%    Oppose our algo to textbook pseudocode.
%    

%A formal proof contains many technical details that are not presented in an informal proof. While dealing with the technical details in a proof causes distraction, leaving out such details makes it hard to follow the proofs.  . So, readers may use this informal proof as a guide to better understand the relevant thoughts in our formal proof. Moreover using the proof language Isar, we made the formal proof readable even for readers that are not familiar with Isabelle.

(TODO: not sure about repeating citation of Cormen) Informal proofs focus on the relevant thoughts by leaving out technical details. However, in a formal proof each step has to be precisely specified as a valid application of some inference rule which makes formal proofs often lengthy and hard to follow. In the case that an interactive theorem prover is used, readers may even need to replay the proof to understand the idea behind it. In Isabelle proof assistant, the Isar proof language bridges the semantic gap between internal notions of proof given by interactive theorem proving systems and an appropriate level of abstraction for user-level work. Isar makes formal proofs readable even for those who are not familiar with Isabelle concepts. We have used such a functionality to present our formal proof of the Ford-Fulkerson method which resembels the informal proof described by Cormen et al.~\cite{CLRS09}.

As an example consider the following proof obtained form Cormen's algorithm book. Let $cf$ be the residual graph corresponding to a flow $f$ in an $s$-$t$ network $c$, and let $f'$ be a flow in $cf$. Consider the definition of the augment function that was described in previous section. Then, the result of augmenting $f$ with $f'$ is another function which satisfies the capacity constraint of a $s$-$t$ flow in $c$.

\emph{\textbf{Proof.}} first observe that if $(u, v) \in E$, then $c_f(v, u) = f(u, v)$. Therefore, we have $f'(v,u) \leq c_f(v, u) = f(u, v)$, and hence
	\begin{align*}
	(f \uparrow f') (u, v) &= f(u, v) + f'(u, v) - f'(v, u)  && \text{(by definition of augment)} \\
	& \geq f(u, v) + f'(u, v) - f(v, u) && \text{(because $f'(v,u) \leq f(u, v)$)} \\
	& = f'(u, v) \\
	& \geq 0.
	\end{align*}
In addition,
	\begin{align*}
	(f \uparrow f') (u, v) &= f(u, v) + f'(u, v) - f'(v, u)  && \text{(by definition of augment)} \\
	& \leq f(u, v) + f'(u, v) && \text{(because flows are nonnegative)} \\
	& \leq f(u, v) + c_f(u, v) &&  \text{(capacity constraint)} \\
	& = f(u, v) + c(u, v) - f(u, v) && \text{(definition of $c_f$)} \\
	& = c (u, v).
	\end{align*}

Following we present the equivalent Isar proof. Note that we have to additionally consider the case that $(u, v) \notin E$. Moreover in line 6, we have used the assumption that there are no parallel edges in a flow network. This important property is not mentioned in Cormen's proof.
\Snippet{augment_flow_presv_cap}


In reference books, it is a convention to use pseudocodes for presenting the algorithms. The abstract version of the algorithm has a clean correctness proof as it only captures the idea behind the algorithm and leaves out the implementation details. We present the Ford-Fulkerson method using the syntax of (TODO: here I mention autoref again, I was not aware that we are not using it anymore. So please correct all the references that I made to autoref tool) Autoref~\cite{La13} tool which is quite similar to pseudocode of the algorithm:
\Snippet{ford_fulkerson_algo}

In the pseudocode of the Ford-Fulkerson method, the path found while evaluating the loop condition is later used for augmenting the flow inside the loop body. The best way for implementing such requirement using syntax of the refinement framework is moving the path lookup inside the loop body. To do so, we modified the while-loop by introducing a break flag. The flag is initially off, and is set if no augmenting path is found in the residual graph.


\section{Refinement in Isabelle/HOL}
When formalizing algorithms, there is often a trade off between verifiability 
and efficiency: An efficient algorithm that uses elaborate data structures and optimizations is often harder to 
verify than a simple but straightforward algorithm. 

The idea of stepwise refinement \cite{Wirth} is to first specify a simple, abstract version of an algorithm, which can be easily verified,
and then refine this algorithm towards an efficient implementation. Here, each refinement step can focus on a single aspect of the algorithm or 
its implementation, independently of the other refinement steps. This results in a greatly increased modularity of proofs. 
Refinement calculi \cite{Back} are used to systematically perform refinement in a Hoare-logic setting, and are particularly well-suited for 
implementation in theorem provers. 

Note that it is important to support nondeterminism in a refinement setting, as this allows to defer 
implementation choices to the later refinement steps. For example, if we want to specify an algorithm 
that returns a shortest path between two nodes of a graph, we do not want to fix a particular shortest path.
Only later, when we decide to implement the specification by, e.g., breadth first search, we fix a particular path.

In Isabelle/HOL, stepwise refinement is supported by the Isabelle Refinement Framework~\cite{LaTu12}. 
It features a refinement calculus for programs phrased in a nondeterminism monad. 
The monad's type is a set of results plus an additional value that indicates a failure:
\begin{lstlisting}
  datatype \<alpha> nres = res \<alpha> set | fail
\end{lstlisting}
The return operation \isai{return x} of the monad describes the single result \isai{x}, and the bind 
operation \isai{bind m f} nondeterministically picks a result from $m$ and executes $f$ on it. 
If either \isai{m = fail}, or \isai{f} may fail for a result in $m$, the bind operation fails.

On nondeterministic results we define the \emph{refinement ordering} by lifting the subset ordering with \isai{fail} being the greatest element.
Intuitively, $m\le m'$ means that $m$ is a refinement of $m'$, i.e., all possible results of $m$ are 
also possible results of $m'$. 
Note that this ordering is a complete lattice, and bind is monotonic. Thus, we can define recursion using a fixed point construction~\cite{Kr10}.
Moreover, we can use the standard Isabelle/HOL constructs for if, let and case distinctions, yielding a fully fledged programming 
language, shallowly embedded into Isabelle/HOL's logic. For simpler usability, we define constants for loop constructs (while, foreach), 
assertions and specifications, and use a Haskell-like do-notation:
\begin{lstlisting}
  assert \<Phi> = if \<Phi> then return () else fail
  spec x. P x = RES {x. P x}
  do {x <- m; f x} = bind m f
  do {m; m'} = bind m (%_. m')
\end{lstlisting}

Correctness of a program $m$ with precondition $P$ and postcondition $Q$ is expressed as \isai{P ==> m <= spec r. Q r}, which
means that, if $P$ holds, $m$ does not fail and all possible results of $m$ satisfy $Q$. Note that we provide different recursion constructs
for partial and total correctness: A nonterminating total correct recursion yields \isai{fail}, while a nonterminating partial correct 
recursion yields \isai|res {}|, which is the least element of the refinement ordering.

The Isabelle Refinement Framework also supports data refinement, changing the representation of results according to a \emph{refinement relation}, 
which relates concrete with abstract results: Given a relation $R$, \isai{\<Down> R m} is the largest set of concrete results that are related to an 
abstract result in $m$ by $R$. If \isai{m=fail}, then also \isai{\<Down> R m = fail}.

Finally, the Isabelle Refinement Framework provides a refinement calculus, which comes with a verification condition 
generator to simplify its usage.

In a typical program development using the Isabelle refinement framework, one first comes up with an initial version $m_0$ of
the algorithm and its specification $P,Q$, and shows \isai{P ==> m_0 <= spec Q}. Then, one iteratively provides refined versions $m_i$ of the algorithm,
proving \isai{$m_i$ <= \<Down>$R_i$ $m_{i-1}$}. Using transitivity and composability of data refinement, one 
gets \isai{P ==> $m_i$ <= \<Down>$R_i\ldots R_1$ spec Q}, showing the correctness of the refined algorithm.

Monotonicity of the standard combinators also allows for modular refinement: Replacing part of a program by a refined version
results in a program that refines the original program.

In Section~\ref{sec:fofu}, we presented the abstract Ford-Fulkerson algorithm as a program in the Isabelle Refinement Framework.
In the next sections, we describe its stepwise refinement down to executable code.


% Each refinement step is shown to preserve correctness
% 
% Refinement techniques \cite{Wirth,Back} structure the verification of an algorithm into
% modular steps, such that one first verifies the abstract algorithmic ideas, and then 
% 
% 
% 
% Purpose: Separation of concerns.
% 
% Short intro on Refinement Framework:
% 
% Programs phrased as error-set monad. With order structure. 
% Refinement: Less possible results (refinement ordering). Error is greatest element (refined by everything).
% 
% Building chains of refinement, at the end there should be a deterministic algorithm (yielding exactly one result), which is
% contained (by transitivity) in the specification.
% 
% Standard combinators (bind, recursion, if, etc) are monotonic wrt. refinement ordering, which allows for modularity:
% Replacing part of a program by a refined version, yields a refined program. 
% 
% Data refinement: Changing the representation of data. Relation between concrete and abstract representation. 
% $\Uparrow m$ is biggest concrete program that refines $m$ wrt.\ relation $R$. 
% 
% The Isabelle Refinement Framework provides the nres-monad, and tools to reason about program refinement, 
% the most important one being a VCG.




\section{Edmonds-Karp Algorithm}
  The general Ford-Fulkerson method can only be shown to terminate if the edge capacities are integer numbers. 
  An improvement over that is the Edmonds-Karp algorithm~\cite{}, which is obtained when always choosing 
  a shortest augmenting path. In this case, it can be shown that only $O(|V||E|)$ augmentations are performed until the
  algorithm terminates, even for real-valued capacities. A shortest augmenting path can be obtained by breadth first search in time $O(|E|)$,
  yielding an overall complexity of $O(|V||E|^2) $ algorithm.
  
  In our formalization, we refine the specification of an augmenting path to a specification of a shortest augmenting path, which immediately yields 
  an abstract version of the Edmonds-Karp algorithm which refines the Ford-Fulkerson method. 
  
  The larger part of the formalization is spent on proving the complexity bound. Note that the refinement framework does not 
  have a notion of computational complexity, so we cannot even define the runtime of an algorithm. However, we can instrument 
  the while-loop of the algorithm with a counter, which is incremented on each iteration, and prove an upper bound on this counter.
  Moreover, this also yields a termination argument for real-valued capacities.\footnote{As we restricted the algorithm 
  to integer valued capacities at the very abstract level, this is actually not used.}
  
  The idea of the complexity proof is as follows: Note that edges that are reverse to edges on a shortest augmenting path cannot lie 
  on a shortest path itself.
  On every augmentation, at least one edge that lies on a shortest path is flipped. Thus, either the length of the shortest path increases,
  or the number of edges that lie on a shortest path decreases. As the length of a shortest path is at most $|V|$, there are no more than $O(|E||V|)$ iterations.

  TODO: Relate to Cormen-proof: Critical edge, removed by augmentation, bound on how often an edge can get critical.  
    Same idea, but yields a more precise bound. On the other hand, we found our proof to be simpler to formalize, and its enough to
    get the $O(|V||E|)$ bound.
  
  Formalizing the above intuitive argument is more tricky than it seems on first glance. 
  While it easy to prove that, in a \emph{fixed graph}, an edge and its reverse cannot both lie on shortest paths, generalizing the argument
  to a graph transformation which may add reverse edges and removes at least original edge, is tricky. Note that a straightforward induction
  on the length of the augmenting path must fail, as after flipping the first edge, the path is no longer augmenting.
  
  We decided to generalize the statement as follows. Assume we have an original graph $cf$ with a shortest augmenting path $p$, and a transformed graph $cf'$ that has been created from
  $cf$ by adding some flipped edges from $p$ and removing some edges from $p$.
  Then, we consider an edge $(u,v) \in p$, a path $p_1$ from $s$ to $v$ in the original graph, and a path $p_2$ from $u$ to $t$ in the transformed graph.
  
  By induction on the number of reversed edges in $p_2$, we 
  show that $|p_1|+|p_2| > |p|$. In the induction step, we use the proof idea for the single flipped edge $(v,u)$, and,
  if $p_2$ contains more flipped edges, we split $p_2$ at the first flipped edge. Then, the initial segment of $p_2$ contains no flipped edges,
  and thus also is a path in the original graph, and we can apply the induction hypothesis.
  
  Proving that augmentation with a shortest augmenting path actually only adds flipped edges, and removes at least one original edge, required some more work,
  but was straightforward and yielded no surprises.
  
  Finally, we phrase the complexity argument as a measure function $m = l*2|E| + n$, where $l$ is the length of the shortest augmenting path,
  and $n$ is the number of edges that lie on any shortest path. We show that $m$ is decreased by the loop body.
  Adding a special case for loop termination (there are no augmenting paths), and observing that $m$ is bounded by $2|V||E|$, yields 
  the desired upper bound on loop iterations. Refinement of the algorithm to add an explicit loop counter, and asserting the upper bound, is then straightforward.
  



    
\section{Refinement to Executable Code}
  \subsection{Using Breadth First Search}

  In the previous sections, we have described the Ford-Fulkerson and the Edmonds-Karp algorithm abstractly,  
  leaving open how to obtain a (shortest) augmenting path. A standard way to find a shortest path in a graph 
  is breadth first search (BFS). Luckily, we had already formalized a BFS algorithm as an example for the refinement framework.
  However, the existing formalization could only compute the minimum distance between two nodes, without returning an actual path.
  We briefly report on our adapted formalization here, which is displayed in Figure~\ref{TODO: Snippet}: 
  TODO: Reference to figure (line numbers?)
  The abstract algorithm keeps track of a set $C$ of current nodes, a set $N$ of new nodes, and a partial map $P$ from already visited nodes to predecessor nodes.
  Initially, only the start node is in $C$ and $N$ is empty. In each iteration of the loop, a node $u\in C$ is picked, and its successors $v$ that have not yet been 
  visited are added to $N$, and $P~v$ is set to $u$. If $C$ becomes empty, $C$ and $N$ are swapped. If the target node is encountered, the algorithm immediately terminates.
  Note that this implementation is a generalized version of the usual queue implementation of BFS: While a queue enforces a complete order on the encountered nodes, our 
  implementation only enforces an ordering between nodes on the current level and nodes on the next level. 

  For the actual algorithm, we wrap this algorithm by a procedure to handle the special case where source and target nodes are the same, and to extract a shortest path 
  from $P$ upon termination. Moreover, we implement the abstract specification of adding the successor nodes of a node by a loop.
  
  Finally, we prove the following theorem: %TODO: Not actually true, the impls are oly done in bfs2!
%     theorem
%       assumes "src∈V" 
%       shows "bfs src dst ≤ (spec p. 
%         case p of None ⇒ ¬conn src dst | Some p ⇒ shortestPath src p dst)"
  
  Obviously, the BFS algorithm refines the specification for obtaining a shortest path.
  Using this to refine the Edmonds-Karp formalization yields a program that algorithmically specifies all major operations.
    
  \subsection{Manipulating Residual Graphs Directly}\label{sec:impl_res_graph}
  In a next step, we observe that the algorithm is phrased in terms of a flow which is updated until it is maximal. However,
  in order to update the flow, the residual graph is used, which is a combination of the current flow and the network.
  Obviously, computing the complete residual graph each time before searching for an augmenting path would be a bad idea. 
  One solution to this problem is to compute the edges of the residual graph on demand from the network and the current flow.
  Although this solution seems to be common, it has the disadvantage that for each edge of the residual graph, two (or even three) 
  edges of the network and the flow have to be accessed. As edges of the residual graph are accessed in the inner loop, during 
  the BFS, these operations are time critical.
  
  Thus, we chose to let the algorithm operate on a representation of the residual graph directly.
  
  The shortest path specification and bottleneck computation are already 
  phrased on residual graphs. What remains is the computation of the initial residual graph (which equals the Network),
  the augmentation of the residual graph, given an augmenting path and its bottleneck capacity, and the computation of the 
  resulting maximum flow from the residual graph, at the end of the algorithm.
  
  In order to refine these operations, we note that the residual graph uniquely determines the flow (and vice versa). 
%         definition (in Network) "flow_of_cf cf e ≡ (if (e∈E) then c e - cf e else 0)"

  For a valid flow $f$, we have that
%         lemma (in NFlow)
%           "flow_of_cf (residualGraph f) = f"
%           by (rule fo_rg_inv)
    
  Thus, augmentation can be expressed directly on the residual graph:
%         "residualGraph (augment (augmentingFlow p)) (u,v) = (
%           if (u,v)∈set p then (residualGraph f (u,v) - bottleNeck p)
%           else if (v,u)∈set p then (residualGraph f (u,v) + bottleNeck p)
%           else residualGraph f (u,v))"
  
  Having these characterizations, a straightforward refinement yields an algorithm that operates on residual graphs.

%   Note: Straightforward, so omitted here!
%   In a next step, we have to specify iterative implementations for finding the bottleneck capacity and doing the augmentation.
%   This is straightforward by folding over the path, exploiting that an augmenting path is simple, and thus never contains an edge 
%   and its flipped edge at the same time. 
      
  \subsection{Computing Successors}
  In order to find an augmenting path, the BFS algorithm has to compute the successors of a node in the residual graph. 
  Although this can be implemented on the capacity matrix by iterating over all nodes, this implementation tends to be inefficient for sparse graphs,
  where we would have to iterate over many nodes just to find that there is no edge.
  
  A common optimization is to pre-compute a map from nodes to adjacent nodes in the network. As an edge in the residual graph is either parallel or reverse to 
  a network edge, it is enough to iterate over the adjacent nodes in the network, and check whether they are actually successors in the residual graph.
  It is straightforward to show that this implementation actually returns the successors in the residual graph:
%         lemma (in RGraph) rg_succ_ref:
%           assumes A: "is_pred_succ ps c"
%           assumes B: "u∈V"
%           shows "rg_succ2 ps cf u ≤ SPEC (λl. (l,cf.E``{u}) ∈ ⟨Id⟩list_set_rel)"

  \subsection{Using Efficient Data Structures}\label{sec:impl_datastructures}
  In a final step, we have to chose efficient data structures for the algorithm. The objects that the algorithm works with are
  capacities, nodes, edges, the graph of the input network, residual graphs, the adjacency map, paths, and the resulting maximum flow. 
  
  We implement capacities as (arbitrary precision) integer numbers. Note that an implementation as fixed precision numbers would also be possible,
  but requires additional checks on the network to ensure that no overflows can happen. 
  % TODO: Can we trick the Java reference implementation on integer overflow?
  
  We implement nodes as natural numbers less than an upper bound $N$, and residual graphs are implemented by their capacity matrices, which, in turn,
  are realized as arrays of size $N*N$ with row-major indexing.
  The adjacency map of the network is implemented as an array of lists of nodes. 
  
  For the BFS algorithm, we implement the sets $C$ and $N$ by lists, and the predecessor map $P$ as an array of \isai{node option}. 
  The successors of a node are represented as a list with no duplicate elements. The resulting path, which is extracted from $P$ upon termination of
  the BFS, is represented as a list of edges. 
  
  The input network of the algorithm is represented as a function mapping network edges to capacities, which is tabulated into an array
  to obtain the initial residual graph.
  
  This allows flexibility in using the algorithm, as any capacity matrix representation can be converted into a function easily, and 
  without loosing efficiency for read-access\footnote{Due to technical limitations of our tools, this function 
  cannot depend on heap content, but it can use the efficient functional and pseudo-functional datastructures provided by the Isabelle Collection Framework.}. 
      
  The output flow of the algorithm is represented as the residual graph. The user can decide how he computes the maximum flow from it. For example,
  in order to compute the maximum flow value, only the outgoing edges of the source node have to be computed, which is typically less 
  expensive than computing the complete flow matrix. The correctness theorem of the algorithm abstractly states how to obtain the maximum flow from the output.

  Note that their is still some optimization potential left: For example, the BFS algorithm computes the predecessor map $P$.
  Then, it iterates over $P$ to get the shortest path into a list. Then, this list is iterated again to compute the 
  bottleneck, and, finally, another iteration over the list is done to do the augmentation. This calls for a deforestation optimization,
  getting rid of the intermediate list altogether, and iterate two times over the predecessor map: A first time to compute the bottleneck, and a second time
  to do the augmentation. Fortunately, iteration over the shortest path does not significantly contribute to the runtime of the algorithm, so that we leave this
  optimization for future work.
  

  Note that we performed the last refinement step using our Sepref tool~\cite{La15}, which provides tool support for data refinements from 
  purely functional programs of the Isabelle Refinement Framework into imperative programs expressed in Imperative/HOL~\cite{BKHEM08}.
  The formalization of this refinement step consists of setting up the mappings between the abstract and concrete data structures,
  and then using the tools provided by Sepref to synthesize the Imperative/HOL programs and their refinement proofs.
  
  Finally, Imperative/HOL provides a heap-exception monad and comes with a setup for the Isabelle code generator to generate efficient imperative programs
  in OCaml, SML, Scala, and Haskell.

  \subsection{Network Checker}
  We also implemented an algorithm that takes as input a list of edges, a source node, and a target node.
  It converts these to a capacity matrix and an adjacency map, and checks whether the resulting graph satisfies our network assumptions.
  We proved that this algorithm returns the correct capacity matrix and adjacency map iff the input describes a valid network,
  and returns a failure value otherwise.
    
  Combining the implementation of the Edmonds-Karp algorithm with the network checker yields our final algorithm,
  for which we can export code, and have proved the theorem: [...Snippet! edmonds-karp-correct]. 
%     theorem edmonds_karp_correct:
%     "<emp> edmonds_karp el s t <λ
%         None ⇒ ↑(¬ln_invar el ∨ ¬Network (ln_α el) s t)
%       | Some (N,fi) ⇒ ∃⇩Af. Network.is_rflow (ln_α el) N f fi * ↑(isMaxFlow (ln_α el) s t f)
%           * ↑(ln_invar el ∧ Network (ln_α el) s t ∧ Graph.V (ln_α el) ⊆ {0..<N})
%     >⇩t"
  Note that this theorem is stated as a Hoare triple, using separation logic assertions. There are no preconditions on the input.
  If the algorithm returns \isai{None}, then the edge list was malformed or described a graph that does not satisfy the network assumptions.
  Here, \isai{ln_invar} describes well-formed edge lists, i.e., edge lists that have no duplicate edges and no edges with zero capacity,
  and \isai{ln_\<alpha>} describes the mapping from (well-formed) edge lists to capacity matrices.
  If the algorithm returns some number $N$ and residual graph $fi$, then the input was a well-formed edge list that describes a valid network.
  Moreover, the returned residual graph describes a flow $f$ in the network, which is maximal. TODO: Swap in thm!
  As there are no other possible return values, this theorem completely describes the correctness of the algorithm. 
  Note that Isabelle/HOL does not have a notion of execution, thus total correctness of the code finally produced by the code 
  generator cannot be defined. However, the program is phrased in a heap-exception monad, thus introducing some (coarse grained) notion of
  computation. On this level, termination can be ensured, and, indeed, the above theorem implies that all the recursions stated by recursion 
  combinators in the monad must terminate. However, it does not guarantee that we have not injected spurious code equations: E.g. $f~x = f~x$,
  is provable by reflexivity, but causes the generated program to diverge.

\section{Benchmarking}
  We have compared the running time of our algorithm against an unverified reference implementation in Java, taken from Sedgewick and Wayne's book on algorithms~\cite{SeWa11}.
  
\begin{figure}[!h]
\centering    
\begin{tikzpicture}[thick,scale=1, every node/.style={scale=1}] %change the scales if you like to reduce the size
	\begin{axis}[
		%title={Benchmark of the Edmonds-Karp algorithm},
	    axis x line*= bottom,
	    axis y line*= left,
	    xmode = log,
	    ymode = log,
	    xlabel = Number of vertices in the network,
	    ylabel = Execution time in milliseconds,
    	smooth,
	    cycle list name = black white,
	    legend style = { 
	    	at={(0.59,0.4)}, 
    	    anchor=north west,
	        draw=black, 
	        fill= white,
        	align=left
    	}
	]
	
	%start data for plotting
		\addplot table {
			1000 204
			2000 1430
			3000 4928
			4000 9299
			5000 23625
		}; \addlegendentry{SML, Sparse};

		\addplot table {
			1000 15
			2000 231
			3000 828
			4000 2145
			5000 4868
		}; \addlegendentry{Java, Sparse};

		\addplot table {
			1000 6575
			1100 8067
			1200 10892
			1300 14070
			1400 17317
		}; \addlegendentry{SML, Dense};

		\addplot table {
			1000 4011
			1100 5487
			1200 7011
			1300 9815
			1400 12156
		}; \addlegendentry{Java, Dense};
	%end data for plotting

	\end{axis}
\end{tikzpicture}
%\caption{Edmonds-Karp benchmark}\label{fig:benchmark}
\end{figure}

  We have done the comparison on randomly generated sparse and dense networks, the sparse networks having a \emph{density} ($\frac{|E|}{|V|(|V| - 1)}$) of $0.02$, and the dense networks having a density of $0.25$. For sparse networks, we varied the number of nodes between $1000$ and $5000$, for dense networks between $1000$ and $1400$.
  The results are shown in Table~\ref{tab:benchmark}. 
  \begin{wrapfigure}{o}{.4\textwidth}
  \vspace{-20pt}
  \begin{center}
  \begin{tabular}{|c|c|c|}
  \hline
  Name        & SML       & Java   \\
  \hline\hline
  1000-sparse & 204       & 15     \\
  2000-sparse & 1430      & 231    \\
  3000-sparse & 4928      & 828    \\
  4000-sparse & 9299      & 2145   \\
  5000-sparse & 23625     & 4868   \\
  1000-dense  & 6575      & 4011   \\
  1100-dense  & 8067      & 5487   \\
  1200-dense  & 10892     & 7011   \\
  1300-dense  & 14070     & 9815   \\
  1400-dense  & 17317     & 12156  \\
  \hline
  \end{tabular}
  \end{center}
  \vspace{-15pt}
  \caption{Dijkstra benchmark}\label{tab:benchmark}
  \vspace{-20pt}
  \end{wrapfigure}
  
  (TODO: Table's caption is incorrect, I think we have enough space to print the whole table with SML, Java as rows and graph size as columns. Also, single lines for separating table looks nicer. I have seen it in your Hopcroft's paper) We observe that, for sparse graphs, the Java implementation is roughly faster by a factor of 6, while for dense graphs, our implementation comes quite close 
  to the Java implementation. We have no full explanation for this phenomenon. Note that the Java implementation operates on flows, while our implementation 
  operates on residual graphs (\cf Section~\ref{sec:impl_res_graph}). Moreover, the Java implementation does not store the augmenting 
  path in an intermediate list, but uses the predecessor map computed by the BFS directly (\cf Section~\ref{sec:impl_datastructures}).

  Finally note that a carefully optimized C++ implementation of the algorithm is only slightly faster than the Java implementation for sparse graphs,
  but roughly one order of magnitude faster for dense graphs. We leave it to future work to investigate this issue, and conclude that we were able to produce
  a reasonably fast verified implementation, without doing any low-level optimizations, but relying on the code generated by the Isabelle/HOL code generator.
  
% 
%   What implementations did we compare: Authors? Where do implementations come from? 
%   [We must convince the reader that we did not intentionally chose bad implementations to compare us against!]
% 
%   Comparsion of algorithms (Modified data structure: Flow vs Flow-like vs ResGraph)
%     Computing successors in BFS: Filter by visited set first.
%       -> Technically more challenging, when computing on flow: 1 access first vs 2 accesses first. On resGraph: No advantage in memory accesses (However, plain array access vs. matrix access)
%     Computing bottleneck during BFS. Saves one iteration over the path, at the cost of one extra memory access per discovered node.
%     For our benchmarks, we observed that the BFS discovers considerably more nodes than the length of the ultimately returned path, 
%     and the overall code was faster with the extra iteration for bottleneck computation.
    
    

\section{Conclusion}
  In this paper, we have presented a verification of the Ford-Fulkerson method, using a stepwise refinement approach.
  Starting with a proof of the Ford-Fulkerson theorem, we have verified the generic Ford-Fulkerson method, then 
  specialized it to the Edmonds-Karp algorithm and proved the upper bound $O(|E||V|)$ for the number of outer loop iterations.
  We then used several refinement steps to derive an efficiently executable implementation of the algorithm, 
  involving a verified breadth first search to obtain shortest augmenting paths. 
  Finally, we added a verified algorithm to check whether the input is a valid network, and generated executable code in SML.
  Our implementation compares well to a reference implementation in Java.
  
  Our formalization has combined several techniques to achieve an elegant and accessible formalization: The Isar-proof language~\cite{}
  allows us to write formal proofs that closely follow informal proofs found in textbooks. This way, we have created a completely rigorous, but 
  still accessible, proof of the Ford-Fulkerson theorem. The Isabelle Refinement Framework allows us to first formalize an algorithm abstractly, 
  and then refine it, in a modular fashion, towards an efficient implementation. This allowed us to present the Ford-Fulkerson method on a level 
  of abstraction that closely resembles pseudocode presentations found in textbooks, and then formally link this presentation to an efficient
  implementation. Moreover, modularity of refinement allowed us to develop the breadth first search algorithm independently, and later link it to the 
  main algorithm. The BFS algorithm can be reused as building block for other algorithms. Also the data structures are re-usable, and in fact, the 
  only new data structure we had to implement where the adjacency matrices. The other data structures where readily available in the growing library 
  of the Sepref tool~\cite{La15} and the Isabelle Collection Framework~\cite{LaTu12}.

  The formalization consists of roughly 8000 lines of proof text, where the graph theory up to the Ford-Fulkerson theorem required 2500 lines.
  The abstract Edmonds-Karp algorithm and its complexity analysis contributes 900 lines, and its implementation (including BFS) another 1500 lines.
  The remaining lines are contributed by the network checker and some auxiliary theories. The development of the theories required roughly 3 men month, a significant amount of this time going into a first, purely functional version of the implementation, which was later dropped in favor of the significantly faster imperative version.
  TODO: Reza, is this estimate OK?
  
  
  \subsection{Related Work}
  We are only aware of one other formalization of the Ford-Fulkerson method conducted in Mizar~\cite{}. 
  It formalizes a fixed instance of the Ford Fulkerson method on a rather abstract level. No code has been generated, 
  nor does it contain a complexity analysis.
  
  Apart from our own work~\cite{La14,NoLa12}, there are several other verifications of graph algorithms, using different techniques. 
  Top-Down
  Bottom-Up: Noschinski verifies a checker for (non-) planarity certificates using a bottom-up approach. Starting at a C implementation,
  the AutoCorres tool~\cite{AutoCorres. What's the standard paper on it?} generates a monadic representation of the program in Isabelle. Further abstractions are applied
  to this representation to hide low-level details like pointer manipulations and fixed size integers~\cite{AutoCorres-Mind-the-Gap}. Finally, a verification condition
  generator is used to prove the abstracted program correct. Note that approach is the opposite direction of ours: While they start at a concrete version of the algorithm and use abstraction steps to eliminate implementation details, we start at an abstract version, and use concretization steps to introduce implementation details.

  Chargu\'eraud also uses a bottom-up approach~\cite{char11} to verify imperative programs written in a subset of OCaml, amongst them a version of Dijkstra's algorithm:
  A verification condition generator generates a \emph{characteristic formula}, which reflects the semantics of the program in the logic of the proof assistant~\cite{}.
  
  \subsection{Future Work}
  Future work includes the formalization of more advanced maximum flow algorithms, like Dinic's algorithm~\cite{Di06} or push-relabel algorithms~\cite{GoTa88}.
  Both, formalizing the abstract theory, and developing efficient implementations will pose some challenges.
  
%   
%   can be performed in two directions. First of all, the abstract theory can be extended to also support more advanced 
%   maximum flow algorithms like push-relabel algorithms~\cite{GoTa88}
%   
%   
%   
%   What do the Coq-Guys have: CFML? Others?
%   
%   
%   XXX, ctd here
%   
%   Unfortunately, it
% 
%   ... and related work
%     Mizar-Formalization: What did they do.
%     
%     History of the formalization: 
%       Impl: From first impl that was hardly able to compute a 100 nodes graph, to current efficient one.
%     
%     Future Work: Dinic. (Actually, our abstract scheme [almost] covers Dinic's algorithm!)
%     
% 
% 
%   
% Contributions
% 
%   Formal proof of mincut maxflow
%   fofo-scheme
%   inst to edmonds karp
%   complexity analysis of edmonds karp
%   refinement down to executable code. Roughly 2.5 times slower than Java. (What about OCaml)
%     + NetCheck
%     
% What shines (its a Pearl)
%   Min-Cut Max-Flow: Textboook like formal reasoning: Comprehensible proof, BUT machine checked
%     (Present one (carefully worked) example in paperI. We could use lemma \isai{augment_flow_presv_cap})
%     
%   Refinement based approach: Fofu-Scheme, instantiation to EdsKa. 
%     +1: Abstract Algo looks almost like pseudo-code you would expect in textbook.
%     +2: Fofu-Scheme proved correct for all aug-path finders. EdsKa is instantiation of it.
%     +3: Modularity: Fofu-scheme and pathfinder developed+proved independently of each other.
%   
%   Down to executable code, plugging in efficient data structures.
%   
% Some minor contributions:
%   Reusable BFS algorithm
%   Imperative matrix data structure (really minor).
%   
%   
%   
%     
% 
% 
% 




\bibliographystyle{abbrv}
\bibliography{root}

\end{document}

